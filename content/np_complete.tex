\documentclass[a4paper,12pt]{article}

% Language and input encoding
\usepackage[utf8]{inputenc}
\usepackage[spanish]{babel}

% Math packages
\usepackage{amsmath, amssymb, amsthm}

% Graphics and hyperlinks
\usepackage{graphicx}
\usepackage{hyperref}

% Page layout
\usepackage{geometry}
\geometry{margin=1in}

% Custom commands
\newtheorem{definition}{Definición}[section]
\newtheorem{theorem}{Teorema}[section]
\newtheorem{lemma}{Lema}[section]
\newtheorem{proposition}{Proposición}[section]

\title{\textbf{Demostraciones de NP-Completitud del problema de partición de grafos}} 
\author{Daniel Machado Pérez \\
Diseño y Análisis de Algoritmos}
\date{}
\begin{document}

\maketitle
\tableofcontents
\newpage




\section{Demostración de que el problema de decisión es NP-Completo}

El problema de decisión se puede definir de la 
siguiente manera: \\

\textbf{Entrada:} Un grafo \( G = (V, E) \), un número entero positivo \( k \), 
una función no negativa de pesos sobre las aristas $w(e)$ 
y un umbral \( M \).\\

\textbf{Pregunta:} Determinar si es posible 
particionar \( G \) en \( k \) componentes conexas de 
manera que la suma de los pesos de las aristas que conectan nodos de 
componentes conexas distintas sea a lo sumo \( M \).\\

Por simplicidad, en lo siguiente llamaremos $P-decision$ a este problema. 

\subsection{Pertenencia a NP}

Para verificar si $P-decision$ pertenece a NP, 
se necesita demostrar que cualquier solución candidata 
puede ser verificada en tiempo polinomial.

Dado un grafo \( G = (V, E) \), una partición de los 
vértices en \( k \) subconjuntos $S_1, S_2, \ldots, S_k$ y un 
umbral \( M \), se puede verificar en tiempo 
polinomial si la solución es válida de la siguiente 
manera:

\begin{itemize}
    \item Comprobar que la partición es válida, es decir, que cada $S_i$ es no vacío, la unión de todos los $S_i$ es $V$ y las intersecciones son vacías $(S_i \cap  S_j = \emptyset$ para $ i \neq j)$. Esto se puede realizar en tiempo \( O(|V|) \) usando estructuras para marcar vértices visitados.
    \item Comprobar que cada subconjunto $S_i$ es una componente conexa. Esto se puede hacer con BFS/DFS en complejidad $O(|V|+|E|)$.
    \item Comprobar que la suma de los pesos de las aristas que cruzan componentes conexas es menor o igual que $M$. Esto se resuelve iterando por todas las aristas, y si una arista involucra vértices de subconjuntos distintos, se suma $w(e)$ al total. Al final se verifica si $w(e) \leq M$. La complejidad es $O(|E|)$.
\end{itemize}

Dado que cada paso se puede realizar en tiempo polinomial, $P-decision$ pertenece a NP.

\subsection{NP-Hardness}
Procederemos a demostrar que $P-decision$ es 
NP-Hard mediante una reducción polinomial desde el 
problema de decisión de \textbf{Clique Máximo}, que es NP-Completo.\\

Recordemos la definición de \textbf{Clique Máximo}:\\
\textbf{Entrada:} Un grafo no dirigido $G = (V, E)$ y $ M \in \mathbb{Z}^{+}$\\
\textbf{Pregunta:} Determinar si existe un clique en $G$ de tamaño mayor o igual que $M$.

\subsubsection{Reducción desde Clique Máximo}

\begin{proof}
    


Consideremos un grafo no dirigido con pesos de $\{0,1\}$ en las aristas. 
Encontrar la mínima configuración de aristas que al ser eliminadas particionan 
el grafo en $k$ componentes conexas, es equivalente a maximizar la cantidad de aristas 
dentro de cada componente conexa. Supongamos que $G$ tiene un clique $H$ de tamaño $M = |V| - (k-1)$.
Entonces el número de aristas entre cualquier vértice $v \notin H$ y $u \in H$ es menor que la 
cantidad de aristas entre vértices dentro de $H$, pues en un clique hay una arista entre cualquier
par de vértices:


\[
    \sum_{u \in H, v \notin H}w(e_{uv}) < \sum_{u \in H, v \in H} w(e_{uv})
\]

Entonces cuando se divide el grafo en $k$ particiones con una cantidad mínima de aristas de cruce entre componentes conexas, el clique $H$ debe ser una de esas componentes.

$\Rightarrow$ El problema de decisión de \textbf{Clique Máximo} puede reducirse en tiempo polinomial a $P-decision$. \\
Como se sabe que \textbf{Clique Máximo} es NP-Completo $\Rightarrow$ $P-decision$ es NP-Hard.

\end{proof}

\subsection{NP-Completitud}

Dado que $P-decision$ pertenece a NP y es NP-Hard, concluimos que es NP-Completo. $\blacksquare$


\section{Demostración de que el problema de optimización es NP-Hard}

El problema de optimización se puede definir de la siguiente manera:\\

\textbf{Entrada:} Un grafo \( G = (V, E) \), un número entero positivo \( k \) y 
una función no negativa de pesos sobre las aristas $w(e)$.\\

\textbf{Salida:} Encontrar una partición de \( G \) en \( k \) componentes conexas de manera que 
la suma de los pesos de las aristas que conectan nodos de componentes conexas distintas es mínima.\\

Por simplicidad, en lo siguiente llamaremos $P-optimization$ a este problema. 

\subsection{NP-Hardness}

\begin{proof}
    Como se conoce que dado un problema, si su variante de decisión es NP-Completo entonces su variante de optimización es NP-Hard, y se demostró que $P-decision$ es NP-Completo $\Rightarrow$ $P-optimization$ es NP-Hard.
\end{proof}


\end{document}
