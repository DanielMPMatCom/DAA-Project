% \documentclass[12pt]{article}
% \usepackage[utf8]{inputenc}
% \usepackage[T1]{fontenc}
% \usepackage[spanish]{babel}
% \usepackage{amsmath,amssymb,amsfonts}
% \usepackage{fullpage}
% \usepackage{hyperref}
% \usepackage{listings}
% \usepackage{xcolor}

% \lstset{
%     basicstyle=\ttfamily\small,
%     numbers=left,
%     numberstyle=\tiny,
%     stepnumber=1,
%     numbersep=10pt,
%     frame=single,
%     breaklines=true,
%     language=C,
%     captionpos=b,
%     tabsize=4,
%     showstringspaces=false,
%     keywordstyle=\color{blue}\bfseries,
%     commentstyle=\color{green!50!black},
%     stringstyle=\color{red},
% }

% \title{daa}
% \author{Leonardo Artiles Montero}
% \date{February 2025}

% \begin{document}

\section{Enunciado del problema}

Dado un árbol con $N$ nodos, calcular para todo $K \leq N$ el mínimo número de aristas a ser
cortadas tal que quede una componente conexa con exactamente $K$ nodos. 

\section{Solución}

Resolvamos el problema utilizando programación dinámica. Rootiemos el árbol $T$, llamemos $T_u$ al subárbol del nodo $u$.
Formulemos nuestra DP de la siguiente forma: 

Sea $dp[u][i]$ la mínima cantidad de aristas que hay que eliminar del árbol para obtener una componente conexa en $T_u$
que contenga al nodo $u$ y sea de tamaño $i$.
Es trivial que teniendo este arreglo rellenado podemos calcular la solución para cada $K$ en $O(N)$.

Veamos cómo calcular esta DP. 
Denotemos $T_u^{j}$ como el árbol $T_u$ pero conteniendo solo sus $j$ primeros hijos y sus subárboles.

Llamemos $dp_j[u][i]$ el valor de $dp[u][i]$ en $T_u^{j}$, lo cual es básicamente habiendo procesado los $j$ primeros hijos de $u$.
La idea es calcular $dp_h[u][i]$ donde $h$ es la cantidad de nodos hijos de $u$, es decir, $h = | \{v_1,v_2,v_3,\dots,v_h\} |$ donde $v_i$ es el hijo $i$-ésimo de $u$$$.

Empezamos con $j=0$, en ese caso $u$ es tomado como una hoja en $T_u^{j}$, así que $dp_j[u][1] = 0$.
Luego computamos los valores de $dp_j[u][i]$ siguiendo la siguiente fórmula:

\[
dp_j[u][i] = \min \left( dp_{j-1}[u][i] + 1, \min_{\substack{a+b=i \\  a, b \geq 1}} (dp_{j-1}[u][a] + dp[v_j][b]) \right)
\]

Lo cual es básicamente tomar el mínimo entre: 
\begin{itemize}
    \item Cortar la arista hacia el $j$-ésimo hijo de $u$. 
    \item El mínimo para todo $a, b$ de tener $a$ nodos en $T_u^{j-1}$ y tener $b$ nodos en el subárbol del $j$-ésimo hijo de $u$.
\end{itemize}

\section{Análisis de complejidad}

\subsection{Cota máxima del algoritmo}

Podríamos establecer una cota máxima de $O(N^3)$ para nuestro algoritmo, ya que para cada nodo $u$,
debemos llenar todos los valores de $i$ en $dp[u][i]$, y para hacer esto tenemos que recorrer todas las posibles
particiones $a, b$ tal que $a + b = i$. Sin embargo, podemos encontrar una mejor cota.

\subsection{Enunciado}

Para un nodo fijo $u$, el número de operaciones necesarias para computar todos los valores de 
$dp[v][:]$ donde $v$ es un nodo que pertenece a $T_u$ es $O(|T_u|^2)$.

\subsection{Demostración}

Demostremos este resultado por inducción sobre la profundidad del árbol $T_u$. 

\begin{itemize}
    \item Caso base: Si $T_u$ es una hoja, el lema se cumple trivialmente, ya que solo necesitamos $O(1)$ operaciones.
    \item Paso inductivo: Supongamos que el nodo $u$ tiene $h$ hijos, denotados como $v_1, v_2, \dots, v_h$, y definamos $a_j = |T_{v_j}|$.
    
    Al fusionar el $j$-ésimo hijo, es decir, al calcular $dp_j[u][:]$, recorremos todos los valores de $dp_{j-1}[u][a]$ y 
    $dp[v_j][b]$ para todo $a$ y $b$. Sin embargo, el valor de $a$ puede ser escogido en $|T_u^{j-1}|$ formas y el de $b$ en 
    $|T_{v_j}|$ formas, lo que nos da un número total de operaciones:

    \[
    O(|T_u^{j-1}| \cdot |T_{v_j}|) = O\left( (1 + a_1 + a_2 + \dots + a_{j-1}) \cdot a_j \right)
    \]

    Por la hipótesis de inducción, calcular todos los valores de $dp$ para los nodos en el subárbol de $u$ tomará:

    \[
    O\left( \sum_{j=1}^{h} a_j^2 \right)
    \]

    Sumando el número de operaciones necesarias para calcular los valores de $dp[v][:]$ para todos los vértices $v$ pertenecientes a $T_u$, obtenemos:

    \[
    O\left( \sum_{i=1}^{h} \sum_{j=1}^{h} a_i \cdot a_j + \sum_{j=1}^{h} a_j + \sum_{j=1}^{h} a_j^2 \right)
    \]

    Como el primer término es una doble sumatoria sobre los tamaños de los subárboles, podemos agrupar los términos para obtener:

    \[
    O\left( \left( 1 + \sum_{j=1}^{h} a_j \right)^2 \right) = O(|T_u|^2)
    \]

\end{itemize}

\subsection{Demostración combinatoria alternativa}

También podemos obtener una demostración combinatoria del enunciado. Al fusionar un subárbol se realizan:

\[
O(|T_u^{j-1}| \cdot |T_{v_j}|)
\]

operaciones. Esto se puede interpretar como el número de pares de nodos que tienen a $u$ como su ancestro común más cercano (LCA).
Como cualquier par de nodos tiene un único LCA, cada par se cuenta una sola vez, lo que nos lleva a una cota total de:

\[
O(N^2)
\]

\section{Demostración de Correctitud}

Sea un árbol \(T\) con \(N\) nodos y sea \(T_u\) el subárbol enraizado en el nodo \(u\). Para cada nodo \(u\) y para cada entero \(i\) en \(1,\dots,|T_u|\), definimos $dp[u][i]$ como el mínimo número de aristas a cortar en  $T_u$ para obtener una componente conexa de tamaño $i$ que contiene a  $u$.
Además, al procesar el nodo \(u\) consideramos sus hijos en un orden fijo y definimos \(T_u^j\) como el árbol formado por \(u\) y los subárboles de sus primeros \(j\) hijos. Sea
\[
dp_j[u][i] = \text{costo mínimo en } T_u^j \text{ para obtener una componente conexa de tamaño } i \text{ que contiene a } u.
\]
El caso base es \(j=0\), donde el único nodo es \(u\), de modo que
\[
dp_0[u][1] = 0,
\]
y para \(i \neq 1\) se establece \(dp_0[u][i] = +\infty\) (o un valor que indique imposibilidad).

Procedemos a demostrar la correctitud mediante doble inducción: primero sobre el número de hijos procesados (índice \(j\)) y luego sobre la estructura del árbol.

\subsection*{1. Inducción sobre la fusión de hijos en \(u\)}

\textbf{Caso base:} Con \(j=0\), el subárbol parcial \(T_u^0\) consiste únicamente en \(u\); por lo tanto,
\[
dp_0[u][1] = 0,
\]
lo cual es correcto.

\textbf{Paso inductivo:}  
Supongamos que para cierto \(j-1\) la tabla \(dp_{j-1}[u][\cdot]\) calcula correctamente el costo mínimo para cada tamaño \(i\) en el subárbol \(T_u^{j-1}\). Sea \(v_j\) el \(j\)-ésimo hijo de \(u\). Al incorporar el subárbol \(T_{v_j}\), se tienen dos opciones:
\begin{enumerate}
    \item \textbf{No incluir \(T_{v_j}\):}  
    Se corta la arista \((u,v_j)\) y se conserva el estado anterior, con un costo adicional de 1:
    \[
    dp_j[u][i] \le dp_{j-1}[u][i] + 1.
    \]
    \item \textbf{Fusionar \(T_{v_j}\):}  
    Se toma una componente en \(T_u^{j-1}\) de tamaño \(a\) (con costo \(dp_{j-1}[u][a]\)) y una componente en \(T_{v_j}\) de tamaño \(b\) (con costo \(dp[v_j][b]\)); al fusionarlas se obtiene una componente de tamaño \(a+b\) con costo:
    \[
    dp_j[u][a+b] \le dp_{j-1}[u][a] + dp[v_j][b].
    \]
\end{enumerate}

Por lo tanto, para cada \(i\) se tiene:
\[
dp_j[u][i] = \min \Biggl\{\, dp_{j-1}[u][i] + 1,\; \min_{\substack{a+b=i \\ a,b\ge1}} \Bigl( dp_{j-1}[u][a] + dp[v_j][b] \Bigr) \,\Biggr\}.
\]
Dado que, por hipótesis inductiva, \(dp_{j-1}[u][\cdot]\) y \(dp[v_j][\cdot]\) son correctos, la fusión produce correctamente \(dp_j[u][\cdot]\) en \(T_u^j\).

Al procesar todos los hijos de \(u\) (es decir, \(j=h\), donde \(h\) es el número total de hijos de \(u\)), se obtiene:
\[
dp_h[u][\cdot] = dp[u][\cdot],
\]
lo que demuestra la corrección del estado \(dp[u][\cdot]\) en \(T_u\).

\subsection*{2. Inducción en la estructura del árbol}

Asumiendo que para cada hijo \(v\) de \(u\) la tabla \(dp[v][\cdot]\) es correcta, el proceso de fusión en \(u\) descrito anteriormente garantiza que \(dp[u][\cdot]\) se calcula correctamente. Por inducción sobre la estructura del árbol, la DP es correcta para todo el árbol \(T\).

\bigskip

\section{Pseudocódigo en \(\boldsymbol{O(N^2)}\)}

A continuación se presenta el pseudocódigo que implementa la solución mediante DFS y fusión de estados. Se aprovecha que, al fusionar el estado del nodo \(u\) (de tamaño \(\text{size}[u]\)) con el del hijo \(v\) (de tamaño \(\text{size}[v]\)), se realizan \(\text{size}[u] \times \text{size}[v]\) operaciones, y gracias a la propiedad combinatoria (cada par de nodos se fusiona una única vez, correspondiente a su LCA) la complejidad total es \(O(N^2)\).

\begin{lstlisting}[caption={Pseudocódigo en \(O(N^2)\)}]
// Constante que representa un valor "infinito"
const INF = infinito;  

// dp[u][i]: Costo minimo para obtener una componente conexa de tamaño i en T_u que contiene a u.
// size[u]: Tamano actual del arreglo dp[u] (inicialmente 1, pues solo se cuenta el nodo u).

procedure DFS(u, parent)
    // Inicializar dp[u] con el caso base: solo el nodo u.
    dp[u] := array[1 ... (tamano maximo posible)] of INF;
    dp[u][1] := 0;
    size[u] := 1;

    // Procesar cada hijo v de u (evitando regresar al padre)
    for each v in vecinos(u) do
        if v == parent then continue;
        DFS(v, u);

        // Preparar un nuevo arreglo para fusionar dp[u] y dp[v]
        newSize := size[u] + size[v];
        newDP := array[1 ... newSize] of INF;

        // Opcion 1: No fusionar v (cortar la arista (u,v))
        for i from 1 to size[u] do
            newDP[i] := min(newDP[i], dp[u][i] + 1);
        end for;

        // Opcion 2: Fusionar el subarbol de v sin cortar (usar dp[v])
        for i from 1 to size[u] do
            for j from 1 to size[v] do
                newDP[i + j] := min(newDP[i + j], dp[u][i] + dp[v][j]);
            end for;
        end for;

        // Actualizar dp[u] y su tamano
        size[u] := newSize;
        dp[u] := newDP;
    end for;
end procedure;

// Funcion principal: se asume que el arbol tiene N nodos.
procedure Main()
    // Se asume que el arbol esta almacenado en una lista de adyacencia: G.
    // Se elige un nodo arbitrario como raiz, por ejemplo, 1.
    DFS(1, NIL);

    // Actualizar la respuesta global para cada tamano K.
    // La solucion final es el minimo dp[u][K] entre todos los nodos u.
    answer := array[1 ... N] of INF;
    for each nodo u de 1 a N do
        for k from 1 to size[u] do
            answer[k] := min(answer[k], dp[u][k]);
        end for;
    end for;

    // Imprimir o retornar la respuesta para cada K = 1, 2, …, N.
end procedure;
\end{lstlisting}

% \end{document}
