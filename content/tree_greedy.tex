\documentclass[12pt]{article}
\usepackage{amsmath, amssymb, amsthm}
\usepackage{geometry}
\geometry{a4paper, margin=1in}
\usepackage{listings}
\usepackage{xcolor}

\newtheorem{lemma}{Lemma}
\newtheorem{proposition}{Proposition}
\newtheorem{corollary}{Corollary}

\lstset{
    basicstyle=\ttfamily,
    columns=fullflexible,
    frame=single,
    breaklines=true,
    postbreak=\mbox{\textcolor{red}{$\hookrightarrow$}\space},
}








\title{Solución de subproblemas con estrategia Greedy}
\author{}
\date{}

\begin{document}

\maketitle

En esta sección se expone la idea de los algoritmos Greedy para resolver el problema en dos casos particulares: cuando el grafo es un árbol y cuando es un bosque.

\section{Caso en que el grafo es un árbol}


Dado un árbol \( T = (V,E) \) con pesos no negativos en sus aristas, la idea es aprovechar la propiedad de que, al remover \( k-1 \) aristas, el árbol se divide en \( k \) componentes conexas. El algoritmo Greedy consiste en:
\begin{enumerate}
    \item Ordenar las aristas del árbol en forma no decreciente según su peso.
    \item Seleccionar las \( k-1 \) aristas de menor peso.
\end{enumerate}
\subsection{Análisis de Complejidad}
\begin{itemize}
    \item Ordenar las aristas requiere \( O(n \log n) \), donde \( n \) es el número de vértices (recordando que un árbol tiene \( n-1 \) aristas).
    \item La selección de las \( k-1 \) aristas es \( O(k) \), lo cual es polinomial.
\end{itemize}
Por lo tanto, el algoritmo Greedy para árboles se resuelve en tiempo polinomial.

\subsection{Demostración de Correctitud}

Recordemos que, en un árbol, al eliminar ciertas aristas se incrementa el número de componentes conexas.

\begin{lemma}[Número de componentes en un árbol]
Si eliminamos todas las aristas de un árbol con \( n \) vértices, es decir, eliminamos \( n-1 \) aristas, obtenemos \( n \) componentes conexas.
\end{lemma}

\begin{proof}
Procedemos por inducción en el número de vértices \( n \).

\textbf{Caso base:}  
Si \( n = 2 \), un árbol tiene exactamente una arista. Al remover esta única arista, el grafo se divide en 2 componentes conexas, lo cual cumple la afirmación.

\textbf{Hipótesis de inducción:}  
Supongamos que para un árbol con \( n = k-1 \) vértices, removiendo \( (k-2) \) aristas obtenemos \( k-1 \) componentes conexas.

\textbf{Paso inductivo:}  
Consideremos un árbol \( T \) con \( k \) vértices. Removamos una arista \( e \) del árbol. Al hacerlo, el árbol se divide en dos subárboles, digamos \( T_1 \) y \( T_2 \), que tienen \( n_1 \) y \( n_2 \) vértices respectivamente, donde \( n_1 + n_2 = k \) y ambos \( n_1, n_2 \ge 1 \).  
Por la hipótesis de inducción, removiendo \( n_1 - 1 \) aristas de \( T_1 \) se obtienen \( n_1 \) componentes conexas y removiendo \( n_2 - 1 \) aristas de \( T_2 \) se obtienen \( n_2 \) componentes conexas.  
En total, removiendo \( 1 + (n_1 - 1) + (n_2 - 1) = n_1 + n_2 - 1 = k - 1 \) aristas, obtenemos \( n_1 + n_2 = k \) componentes conexas.  
\end{proof}

\begin{lemma}
Sea \( T = (V, E) \) un árbol con pesos no negativos en sus aristas, y sea la lista de aristas ordenada de forma no decreciente según sus pesos:
\[
w(e_1) \le w(e_2) \le \dots \le w(e_{n-1}).
\]
Entonces, el conjunto de aristas de menor peso que separa el árbol en \( k \) componentes conexas está dado por:
\[
C = \{ e_1, e_2, \dots, e_{k-1} \}.
\]
\end{lemma}

\begin{proof}
Procedemos por contradicción. Supongamos que existe otro conjunto \( C' \) de \( k-1 \) aristas tal que el peso total \( w(C') \) es menor que \( w(C) \).  
Sin embargo, al remover \( k-1 \) aristas de \( T \), por el resultado del lema anterior, el árbol se divide en \( k \) componentes conexas.  
Como \( C \) está formado por las \( k-1 \) aristas de menor peso, cualquier otro conjunto \( C' \) de \( k-1 \) aristas necesariamente tendrá un peso total mayor o igual a \( w(C) \).  
Esto contradice la hipótesis de que \( w(C') < w(C) \), por lo que \( C \) debe ser el conjunto deseado.
\end{proof}

\begin{corollary}
Sobre árboles, el problema se puede resolver en tiempo polinomial, ya que basta con ordenar las aristas (lo cual se puede hacer en tiempo \( O(n \log n) \)) y seleccionar las \( k-1 \) aristas de menor peso.
\end{corollary}

\subsection{Pseudocódigo}

% \begin{verbatim}
% Entrada: Árbol T = (V, E) con pesos no negativos, entero k
% Salida: Conjunto de aristas C que forma un corte mínimo k

% 1. Ordenar las aristas E de T en forma no decreciente según su peso: 
%       w(e1) <= w(e2) <= ... <= w(e_{|V|-1})
% 2. Seleccionar las primeras k-1 aristas de la lista ordenada y asignarlas a C.
% 3. Retornar C.
% \end{verbatim}


\section{Caso en que el grafo es un bosque}
Para un bosque \( F = (V,E) \) con \( t \) componentes conexas, la idea es similar. Se debe notar que, para obtener una partición en \( k \) componentes, es necesario remover \( k-t \) aristas, ya que el bosque ya cuenta con \( t \) componentes. El algoritmo Greedy consiste en:
\begin{enumerate}
    \item Ordenar todas las aristas del bosque en forma no decreciente según su peso.
    \item Seleccionar las \( k-t \) aristas de menor peso.
\end{enumerate}


\subsection{Análisis de Complejidad}
\begin{itemize}
    \item Ordenar las aristas requiere \( O(|E| \log |E|) \); dado que el bosque tiene a lo sumo \( |V|-1 \) aristas, se cumple que es polinomial.
    \item La selección de las \( k-t \) aristas es \( O(k) \).
\end{itemize}
Por lo tanto, este algoritmo también se ejecuta en tiempo polinomial.

\subsection{Demostración de Correctitud}

Consideremos ahora un bosque, es decir, un grafo disconexo cuyos componentes son árboles.

\begin{proposition}
Sea \( t \) el número de componentes conexas en un bosque \( F = (V, E) \). Para cualquier \( n \ge t \), si eliminamos \( (n-t) \) aristas del bosque, obtenemos \( n \) componentes conexas.
\end{proposition}

\begin{proof}
Procedemos por inducción en \( n \).

\textbf{Caso base:}  
Si \( n = t \), no eliminamos ninguna arista, y el bosque ya tiene \( t \) componentes, lo que cumple la afirmación.

\textbf{Hipótesis de inducción:}  
Supongamos que para \( n = k-1 \) (con \( k-1 \ge t \)), removiendo \( (k-1 - t) \) aristas se obtienen \( k-1 \) componentes conexas.

\textbf{Paso inductivo:}  
Consideremos un bosque \( F \) y supongamos que removiendo \( (k-1-t) \) aristas se obtienen \( k-1 \) componentes conexas.  
Ahora, removamos una arista adicional de uno de los árboles resultantes. Por el lema aplicado a árboles, al remover una arista de ese árbol se incrementa el número de componentes en 1.  
Así, en total, removiendo \( (k-1-t) + 1 = k-t \) aristas, obtenemos \( k \) componentes conexas.  
\end{proof}

\begin{lemma}
Sea \( F = (V, E) \) un bosque con pesos no negativos en sus aristas, y sean \( t \) el número de componentes conexas de \( F \) y las aristas ordenadas por la función de peso de forma no decreciente:
\[
w(e_1) \le w(e_2) \le \dots \le w(e_{|V|}).
\]
Entonces, el conjunto de aristas de menor peso cuya eliminación divide el bosque en \( k \) componentes conexas está dado por:
\[
C = \{ e_1, e_2, \dots, e_{k-t} \}.
\]
\end{lemma}

\begin{proof}
Supongamos, para obtener una contradicción, que existe un conjunto \( C' \) de aristas con \( |C'| < k-t \) y \( w(C') < w(C) \) que, al removerlas, divide el bosque en \( k \) componentes conexas.  
Sin embargo, según la proposición anterior, remover menos de \( k-t \) aristas en un bosque con \( t \) componentes produce menos de \( k \) componentes conexas.  
Esto contradice la definición de $C'$.  
Por lo tanto, \( C \) es el conjunto de aristas de menor peso cuya eliminación divide \( F \) en \( k \) componentes conexas.
\end{proof}


\subsection{Pseudocódigo}

% \begin{verbatim}
% Entrada: Bosque F = (V, E) con pesos no negativos, entero k, 
%          donde t es el número de componentes conexas en F.
% Salida: Conjunto de aristas C que forma un corte mínimo k en F

% 1. Calcular t, el número de componentes conexas de F.
% 2. Ordenar las aristas E de F en forma no decreciente según su peso:
%       w(e1) <= w(e2) <= ... <= w(e_{|V|-?})
% 3. Seleccionar las primeras k-t aristas de la lista ordenada y asignarlas a C.
% 4. Retornar C.
% \end{verbatim}







\end{document}
