\documentclass[12pt]{article}
\usepackage{amsmath, amssymb, amsthm}

\newtheorem{lemma}{Lemma}
\newtheorem{proposition}{Proposition}
\newtheorem{corollary}{Corollary}
\usepackage{geometry}
\geometry{a4paper, margin=1in}

\title{Demostración de correctitud de un algoritmo Greedy para resolver el Problema}
\author{}
\date{}

\begin{document}

\maketitle






\section{Para el caso en que el grafo es un árbol}

Recordemos que, en un árbol, al eliminar ciertas aristas se incrementa el número de componentes conexas.

\begin{lemma}[Número de componentes en un árbol]
Si eliminamos todas las aristas de un árbol con \( n \) vértices, es decir, eliminamos \( n-1 \) aristas, obtenemos \( n \) componentes conexas.
\end{lemma}

\begin{proof}
Procedemos por inducción en el número de vértices \( n \).

\textbf{Caso base:}  
Si \( n = 2 \), un árbol tiene exactamente una arista. Al remover esta única arista, el grafo se divide en 2 componentes conexas, lo cual cumple la afirmación.

\textbf{Hipótesis de inducción:}  
Supongamos que para un árbol con \( n = k-1 \) vértices, removiendo \( (k-2) \) aristas obtenemos \( k-1 \) componentes conexas.

\textbf{Paso inductivo:}  
Consideremos un árbol \( T \) con \( k \) vértices. Removamos una arista \( e \) del árbol. Al hacerlo, el árbol se divide en dos subárboles, digamos \( T_1 \) y \( T_2 \), que tienen \( n_1 \) y \( n_2 \) vértices respectivamente, donde \( n_1 + n_2 = k \) y ambos \( n_1, n_2 \ge 1 \).  
Por la hipótesis de inducción, removiendo \( n_1 - 1 \) aristas de \( T_1 \) se obtienen \( n_1 \) componentes conexas y removiendo \( n_2 - 1 \) aristas de \( T_2 \) se obtienen \( n_2 \) componentes conexas.  
En total, removiendo \( 1 + (n_1 - 1) + (n_2 - 1) = n_1 + n_2 - 1 = k - 1 \) aristas, obtenemos \( n_1 + n_2 = k \) componentes conexas.  
\end{proof}

\begin{lemma}[Corte mínimo \( k \) en un árbol]
Sea \( T = (V, E) \) un árbol con pesos no negativos en sus aristas, y sea la lista de aristas ordenada de forma no decreciente según sus pesos:
\[
w(e_1) \le w(e_2) \le \dots \le w(e_{n-1}).
\]
Entonces, el conjunto de aristas de menor peso que separa el árbol en \( k \) componentes conexas está dado por:
\[
C = \{ e_1, e_2, \dots, e_{k-1} \}.
\]
\end{lemma}

\begin{proof}
Procedemos por contradicción. Supongamos que existe otro conjunto \( C' \) de \( k-1 \) aristas tal que el peso total \( w(C') \) es menor que \( w(C) \).  
Sin embargo, al remover \( k-1 \) aristas de \( T \), por el resultado del lema anterior, el árbol se divide en \( k \) componentes conexas.  
Como \( C \) está formado por las \( k-1 \) aristas de menor peso, cualquier otro conjunto \( C' \) de \( k-1 \) aristas necesariamente tendrá un peso total mayor o igual a \( w(C) \).  
Esto contradice la hipótesis de que \( w(C') < w(C) \), por lo que \( C \) debe ser el conjunto deseado.
\end{proof}

\begin{corollary}
Sobre árboles, el problema se puede resolver en tiempo polinomial, ya que basta con ordenar las aristas (lo cual se puede hacer en tiempo \( O(n \log n) \)) y seleccionar las \( k-1 \) aristas de menor peso.
\end{corollary}

\section{Para el caso en que el grafo es un bosque}

Consideremos ahora un bosque, es decir, un grafo disconexo cuyos componentes son árboles.

\begin{proposition}
Sea \( t \) el número de componentes conexas en un bosque \( F = (V, E) \). Para cualquier \( n \ge t \), si eliminamos \( (n-t) \) aristas del bosque, obtenemos \( n \) componentes conexas.
\end{proposition}

\begin{proof}
Procedemos por inducción en \( n \).

\textbf{Caso base:}  
Si \( n = t \), no eliminamos ninguna arista, y el bosque ya tiene \( t \) componentes, lo que cumple la afirmación.

\textbf{Hipótesis de inducción:}  
Supongamos que para \( n = k-1 \) (con \( k-1 \ge t \)), removiendo \( (k-1 - t) \) aristas se obtienen \( k-1 \) componentes conexas.

\textbf{Paso inductivo:}  
Consideremos un bosque \( F \) y supongamos que removiendo \( (k-1-t) \) aristas se obtienen \( k-1 \) componentes conexas.  
Ahora, removamos una arista adicional de uno de los árboles resultantes. Por el lema aplicado a árboles, al remover una arista de ese árbol se incrementa el número de componentes en 1.  
Así, en total, removiendo \( (k-1-t) + 1 = k-t \) aristas, obtenemos \( k \) componentes conexas.  
\end{proof}

\begin{lemma}
Sea \( F = (V, E) \) un bosque con pesos no negativos en sus aristas, y sean \( t \) el número de componentes conexas de \( F \) y las aristas ordenadas por la función de peso de forma no decreciente:
\[
w(e_1) \le w(e_2) \le \dots \le w(e_{|V|}).
\]
Entonces, el conjunto de aristas de menor peso cuya eliminación divide el bosque en \( k \) componentes conexas está dado por:
\[
C = \{ e_1, e_2, \dots, e_{k-t} \}.
\]
\end{lemma}

\begin{proof}
Supongamos, para obtener una contradicción, que existe un conjunto \( C' \) de aristas con \( |C'| < k-t \) y \( w(C') < w(C) \) que, al removerlas, divide el bosque en \( k \) componentes conexas.  
Sin embargo, según la proposición anterior, remover menos de \( k-t \) aristas en un bosque con \( t \) componentes produce menos de \( k \) componentes conexas.  
Esto contradice la definición de corte mínimo \( k \), que debe separar el bosque en exactamente \( k \) componentes.  
Por lo tanto, \( C \) es el conjunto de aristas de menor peso cuya remoción divide \( F \) en \( k \) componentes conexas.
\end{proof}




\end{document}
