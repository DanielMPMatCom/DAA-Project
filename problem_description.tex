\documentclass[a4paper, 12pt]{article}

% Paquetes necesarios
\usepackage[utf8]{inputenc}  % Soporte UTF-8
\usepackage[spanish]{babel} % Idioma español
\usepackage{amsmath}        % Paquete matemático
\usepackage{amssymb}        % Símbolos matemáticos adicionales
\usepackage{geometry}       % Ajuste de márgenes
\geometry{left=2.5cm, right=2.5cm, top=3cm, bottom=3cm}

\title{\textbf{Optimización de Redes de Distribución Energética}}
\author{}
\date{}

\begin{document}

\maketitle

\section*{Contexto práctico}
En una red de distribución energética de la UNE, 
las subestaciones están interconectadas mediante 
una red de líneas de transmisión. Cada línea tiene 
un costo de mantenimiento (ponderación) y una 
capacidad de transferencia energética. 
Cuando ocurre un fallo en una línea de transmisión 
(nada frecuente, por cierto), 
la red debe reorganizarse dinámicamente (muy dinámico siempre) 
para minimizar los costos de reparación y garantizar 
que el suministro energético a las regiones más 
críticas sea continuo (jaja yeah sure).

El jefe de la UNE desea dividir la red en \( k \) 
grupos independientes (subredes autónomas) durante 
situaciones de mantenimiento o emergencias, 
asegurándose de que la desconexión sea lo menos 
costosa posible (en términos de la suma de los 
costos de las líneas cortadas).

Adicionalmente, al jefe de la UNE le llegó de las altas esferas
la orientación de resolver otra tarea. 
Se quiere garantizar que existan zonas priorizadas que 
no sufran cortes de electricidad por su relevancia. 
Estas zonas pueden incluir hospitales, instituciones importantes como el ICRT, 
residencias de mandatarios (ya de paso jiji), entre otras. 
Por ello, se plantea la necesidad de determinar la 
cantidad mínima de aristas que deben removerse para 
obtener al menos una componente conexa con 
exactamente \( t \) nodos, representando estas zonas 
priorizadas.

\section*{Formalización del problema}
Dado un grafo ponderado \( G = (V, E) \), donde:
\begin{itemize}
    \item \( V \): Conjunto de nodos que representan subestaciones.
    \item \( E \): Conjunto de aristas que representan líneas de transmisión, cada una con un costo \( w(e) \) asociado.
    \item \( k \): Número de subredes requeridas.
    \item \( S = \{s_1, s_2, \dots, s_k\} \subseteq V \): Subestaciones críticas que deben quedar separadas entre las \( k \) subredes.
    \item \( t \): Tamaño de la componente priorizada.
\end{itemize}

\noindent \textbf{Problemas}:
\begin{enumerate}
    \item Encontrar un conjunto mínimo de aristas \( E' \subseteq E \) tal que al eliminar \( E' \):
    \begin{enumerate}
        \item \( G \) se divide en \( k \) componentes conexos.
        \item Cada subestación crítica \( s_i \) pertenece a una componente diferente.
        \item La suma de los costos de las aristas eliminadas \( \sum_{e \in E'} w(e) \) es mínima.
    \end{enumerate}
    \item Determinar la cantidad mínima de aristas a remover del grafo \( G \) tal que quede al menos una componente conexa con exactamente \( t \) nodos.
\end{enumerate}

\section*{Estrategias de solución}
\begin{enumerate}
    \item \textbf{Demostración de NP-Complejidad}: 
    Demostrar que la primera parte del problema es \textbf{NP-completo}, utilizando una reducción al problema \textit{Max-Clique}.
    \item \textbf{Reducción a árboles y solución greedy}: 
    Reducir la primera parte del problema a árboles y diseñar una solución greedy con complejidad \( O(n \log n) \).
    \item \textbf{Algoritmos de aproximación con flujo}: 
    Proponer algoritmos de aproximación basados en técnicas de flujo para resolver la primera parte del problema en grafos generales.
    \item \textbf{Reducción de la segunda parte del problema a árboles}: 
    Reducir la segunda parte del problema a árboles y resolverlo utilizando programación dinámica.
\end{enumerate}


\end{document}